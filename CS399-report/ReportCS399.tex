\documentclass[11pt,a4paper,oldfontcommands]{memoir}
\usepackage[utf8]{inputenc}
\usepackage[T1]{fontenc}
\usepackage{microtype}
\usepackage[dvips]{graphicx}
\usepackage{xcolor}
\usepackage{times}

\usepackage[
breaklinks=true,colorlinks=true,
%linkcolor=blue,urlcolor=blue,citecolor=blue,% PDF VIEW
linkcolor=black,urlcolor=black,citecolor=black,% PRINT
bookmarks=true,bookmarksopenlevel=2]{hyperref}

\usepackage{geometry}
% * <somanath.tripathy@gmail.com> 2018-03-11T09:18:17.948Z:
%
% > }
%
% ^.
% PDF VIEW
% \geometry{total={210mm,297mm},
% left=25mm,right=25mm,%
% bindingoffset=0mm, top=25mm,bottom=25mm}
% PRINT
\geometry{total={210mm,297mm},
left=20mm,right=20mm,
bindingoffset=10mm, top=25mm,bottom=25mm}

\OnehalfSpacing
%\linespread{1.3}

%%% CHAPTER'S STYLE
%\chapterstyle{bianchi}
%\chapterstyle{ger}
%\chapterstyle{madsen}
%\chapterstyle{ell}
%%% STYLE OF SECTIONS, SUBSECTIONS, AND SUBSUBSECTIONS
\setsecheadstyle{\Large\bfseries\sffamily\raggedright}
\setsubsecheadstyle{\large\bfseries\sffamily\raggedright}
\setsubsubsecheadstyle{\bfseries\sffamily\raggedright}


%%% STYLE OF PAGES NUMBERING
%\pagestyle{companion}\nouppercaseheads 
%\pagestyle{headings}
%\pagestyle{Ruled}
\pagestyle{plain}
\makepagestyle{plain}
\makeevenfoot{plain}{\thepage}{}{}
\makeoddfoot{plain}{}{}{\thepage}
\makeevenhead{plain}{}{}{}
\makeoddhead{plain}{}{}{}


%\maxsecnumdepth{subsection} % chapters, sections, and subsections are numbered
\maxtocdepth{subsection} % chapters, sections, and subsections are in the Table of Contents


%%%---%%%---%%%---%%%---%%%---%%%---%%%---%%%---%%%---%%%---%%%---%%%---%%%

\begin{document}

%%%---%%%---%%%---%%%---%%%---%%%---%%%---%%%---%%%---%%%---%%%---%%%---%%%
%   TITLEPAGE
%
%   due to variety of titlepage schemes it is probably better to make titlepage manually
%
%%%---%%%---%%%---%%%---%%%---%%%---%%%---%%%---%%%---%%%---%%%---%%%---%%%
\thispagestyle{empty}

{%%%
\sffamily
\centering
\Large

~\vspace{\fill}
\centering{
{\huge 
Report title: max 20 words (short \& descriptive)
}

\vspace{2.5cm}

{\LARGE
Author  \\ %avoid abbreviations
Roll No. \\
Email: xyz@iitp.ac.in
\\
%{\LARGE
%Author 2 name} \\ %avoid abbreviations
%{Roll No.\\
%Email 
}

\vspace{3.5cm}

"CS399: Seminar Report"\\
Spring 2018

\vspace{3.5cm}
Department of Computer Sc. and Eng.\\
Indian Institute of Technology Patna

\vspace{\fill}

%

%%%
}%%%
}
\cleardoublepage
%%%---%%%---%%%---%%%---%%%---%%%---%%%---%%%---%%%---%%%---%%%---%%%---%%%
%%%---%%%---%%%---%%%---%%%---%%%---%%%---%%%---%%%---%%%---%%%---%%%---%%%

%\tableofcontents*

\clearpage

%%%---%%%---%%%---%%%---%%%---%%%---%%%---%%%---%%%---%%%---%%%---%%%---%%%
%%%---%%%---%%%---%%%---%%%---%%%---%%%---%%%---%%%---%%%---%%%---%%%---%%%
\begin{abstract}


Body of abstract (summary of contributions and/or results; approx. 1 line per page)

	\vspace{1.5cm}
	\textbf{Keywords}: " followed by 5 to 10 keywords and key phrases describing the content
\end{abstract}

\section{Introduction}

%\section{First section}
For the Seminar Report, you should understand an important issue in any area related to Computer Science.  Nice if you tried to prove an extension of the existing work. You will write a report describing the research papers/  articles you read, what you tried to do, and any results, in the format of a conference paper. 4-6 pages length. 

Give background on the topic (provide context and include references on prior work), justify your interest
in the topic, prepare the readers for what they will find in later sections, and summarize (in a few
sentences) the main findings and/or contributions. This section must be kept short about 1 t0 1.5 pages. 

\section{Contribution:  Place a suitable name based on the contribution}
About 1.5- 2 pages Body of the paper dealing with various aspects of the investigation as appropriate; e.g., theory, applications, design issues,
tradeoffs, evaluation, experiments
\section {Discussion}
Here you write your views, comparisons with other methods or approaches. You should
compare, criticize, and generally leave your personal mark on the paper. 
We analyzed the work  in \cite{Oru:2013} in about 1 page.

%\section{Second section}
%Body of second section

%\subsection{subsection}

%Body of subsection 

\section{Conclusion}

Give a brief summary (in a few sentences) of what has been presented and/or accomplished. Emphasize
the advantages and disadvantages of the proposed approach, technique, or design. Discuss possible
extensions of the work and any interesting/open problem that you can envisage.

\bibliographystyle{unsrt}
\bibliography{sample}



\end{document}

